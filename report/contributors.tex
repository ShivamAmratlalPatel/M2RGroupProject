\section{Contributors}\label{sec:contributors}

\subsection{Adhavan Sashikumar}\label{subsec:adhavan-sashikumar}
Most of my work in this project has come in the report, and the rest with coding algorithms into our Python environments and producing plots.
I have written the all the sections on Thompson Sampling (other than its pseudocode), including the explanation of the algorithm, its regret, its strengths and weaknesses and the situations in which it can be best utilised.
I also wrote the section on regret in the introduction, to explain how we measure the strength of an algorithm.
I have also researched and written on the other strategies;
I wrote the sections explaining the random strategy and epsilon greedy algorithms as well as the 'Analysis of Simulations' segments, whilst adding to the sections on Strengths and Weaknesses for the other strategies.
In addition, I have helped to write the algorithms for Thompson sampling, and produced the plot for Figure 7 in our Python environment.
Furthermore, I added a section to the UCB algorithm titled 'Best conditions for UCB algorithm', which I then wrote on.

\subsection{Diandian Chen}\label{subsec:diandian-chen}
My key contribution in this project has come in the report.
I have written all the UCB strategies  under the stationary strategies part including Hoeffding's Inequality, UCB1 and KL-UCB as well as their strengths and weaknesses in the report.
I also wrote the section on GLR-klUCB as a non-stationary strategy with its strengths and weaknesses.
Furthermore, I advised to use latex beamer to generate the slides for presentation then we can get the slides together conveniently and use latex mathematical labels directly.
\subsection{Rui Tang}\label{subsec:rui-tang}
My contribution in this project is in this report.
\newline I have mainly written the epsilon-first strategy under the stationary strategies part and also discovered some strategies content under the non-stationary strategies part, including sections of the Discounted-UCB, Sliding-window UCB, as well as  some contents of the Limited-memory DSEE algorithm, and put them on the report with its principles, strengths and weaknesses.

\subsection{Shivam Patel}\label{subsec:shivam-patel}
My key contribution in this project was programming the environment and simulations.
Coming into the project, I had completed various other projects and hackathons which required programming skills so was excited when this opportunity presented itself.
I was also familiar with git and implemented the use of this both for our codebase, and our report within our group to ensure an effective workflow.
Furthermore, from my experience of latex, I set up the report, referencing methods and taught our group how to use a .bib file and cite to accurately reference work and ensure consistency within the group.
I also took the role of 'group leader' on and delegated tasks to each member of the group and ensured we were working at an efficient pace, which was important in such a short timeline.

\subsection{Xinqi Yao}\label{subsec:xinqi-yao}
My key contribution in this project has come in the report. 
I have written part of the epsilon-greedy strategy, especially epsilon-first strategy, as well as its strengths and weaknesses in the report.
Moreover, I wrote the sliding-window UCB algorithm under the non-stationary strategy section and compared it with Discounted UCB to summarise its characteristics.
