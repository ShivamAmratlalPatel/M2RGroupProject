\section{Non-stationary Strategies}\label{sec:non-stationary-strategies}
The strategies we have explained above work well when the underlying distributions for each arm don't change, but what happens when they do?
In this section, we go through strategies for different non-stationary distribution scenarios, and explain their strengths and weaknesses.

\subsection{Discounted UCB}\label{subsec:discounted-ucb}
Discounted UCB (D-UCB), which was proposed by \citep{kocsis2006discounted}, is a way specifically implying in local channels due to the varying time of local environment.
The principle is adding a discount factor, which is geometric moving average over the samples, to the UCB Tuned (UCBT) algorithm, and the average reward is given by
\begin{equation*}
    \hat{z}_i\left(j\right)=\frac{\sum_{t=1}^T\gamma_i^{T-t}x_i\left(t\right)}{\hat{n}_i\left(j\right)}\mathfrak{1}_{s\left(t\right)=i}, \quad
    {\hat{n}_i\left(j\right)}=\sum_{t=1}^T\gamma_i^{T-t}\mathfrak{1}_{s\left(t\right)=i},
\end{equation*}
where the $\gamma_i \in[0,1]$ represents the change rate of i and is also the discount factor.
The algorithm is also shown to achieve the upper-bounded regret by $O\left(\sqrt{\gamma_{i}T}\log{T}\right)$\citep{garivier2008upper}.

\subsection{Sliding-Window UCB}\label{subsec:sliding-window-ucb}
The Sliding-Window UCB (SW-UCB) algorithm is similar to the D-UCB, but it updates observation in UCB1 algorithm and consider average reward only within a window with fixed length of l, where l decreases when the change rate of environment increases.
The maximum regret of this algorithm is proven to be $O\left(\sqrt{\gamma_{i}T\log{T}}\right)$\citep{garivier2008upper}.
\\Compared with D-UCB, the computational complexity of SW-UCB is linear in time and does not involve $j$.
However, SW-UCB requires to store the last $j$ actions and rewards at each time $t$ in order to efficiently update ${\hat{n}_i}$ as well as ${\hat{z}_i}$~\citep{garivier2008upper}.

\subsection{GLR-klUCB}\label{subsec:glr-klucb}
GLR-klUCB is a algorithm which combines an efficient bandit algorithm klUCB, with an parameter- free, change-point detector, the Bernoulli Generalized Likelihood Ratio Test.
Regret is upper-bounded by $ O(\Upsilon_T \sqrt{T \log(T)} )$ if the number of changed points $\Upsilon_T $ is unknown and $ O(\sqrt{\Upsilon_T T \log(T)} )$ if  $\Upsilon_T $ is known.\citep{Besson2019}


\\For a simple model with global changes, choosing parameter $\alpha$ and $\delta $ as $\sqrt{\frac{\logT}{T}}$ and $1/\sqrt{T}$ respectively, the regret becomes $O(\frac{K}{(\delta^change)^2} \Upsilon_T \sqrt{T \log(T)}+\frac{(K-1)}{\delta^opt} \Upsilon_T \log(T))$. Here $\delta^opt$ denote the smallest value of a sub-optimality gap on one of the stationary segments, $\delta^change$ be the smallest magnitude of any change point on any arm and K is the number of arms.
\citep{Besson2019}


\\This strategy is efficient for the piece-wise stationary bandit problem. This actively adaptive method attains state-of-the-art regret upper-bounds when tuned with a prior knowledge of the number of changes ΥT , but without any other prior knowledge on the problem.\citep{Besson2019}

\subsection{ Limited-Memory DSEE (LM-DSEE)}
The LM-DSEE algorithm comprises interleaving epochs of exploration and exploitation. In the k-th exploration epoch, each arm is sampled L(k)=$\lceil\gamma\ln({k^{\rho}lb})\rceil$

